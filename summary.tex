\documentclass[document.tex]{subfiles}
\begin{document}

\section*{Summary: Explaining capital structure \protect\\ - Verbeek p.383-388}
This chapter in the book serves as a research example for comparing the properties and results of OLS, first differences, the within, Anderson Hsisao
instrumental variable and Arellano Bond GMM estimators. \\
The theoretical model underlying this analysis is a partial-adjustment model. For the empirical analysis a dynamic panel data model is estimated using the different estimators mentioned earlier and the results provided by the different methods are compared.\\
The research question raised by Verbeek is: ''Is it possible to properly investigate the 
deviations and the adjustment speed of the debt ratio $MDR_{it}$ and the target debt ratio $MDR^*_{it}$ using firm characteristics as proxies for the potential costs and benefits of a firms debt structure.'' This question continues a broadly discussed topic, the optimal capital structure (equity vs debt) of the firm. First work was done by famous Modigliani and Miller in 1958 where they showed that in friction less markets the capital structure is irrelevant. In the real world however, we have to take frictions into account. Especially if we are doing an empirical investigation.

\end{document}

